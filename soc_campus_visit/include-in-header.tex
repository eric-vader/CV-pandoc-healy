
% \usepackage[linesnumbered,ruled,vlined,algochapter]{algorithm2e}
% \usepackage{lscape}
% \usepackage{xspace}
% \usepackage{xfrac}
% \usepackage{enumitem}
% \usepackage{amsthm}

% /Centering 
\usepackage{ragged2e}
\usepackage{tcolorbox}
\usepackage{amsmath}
\usepackage{mathtools}
\usepackage{subfig}
\usepackage{xspace}

\usepackage{booktabs}
\usepackage{multirow}
\usepackage{rotating}
\usepackage{makecell}
\usepackage[edges]{forest}
\usepackage{stackengine}

\usetikzlibrary{decorations.markings}
\newsavebox\mybox
\savebox\mybox{%
  \tikz{
    \draw[ultra thick,red] (-4pt,-4pt) -- (4pt,4pt);
    \draw[ultra thick,red] (-4pt,4pt) -- (4pt,-4pt);
  }%
}  

\tikzset{
prune/.style={
  decoration={
   markings,
   mark=at position 0.5 with \node {\usebox\mybox};
  },
  postaction=decorate
  }
}
\forestset{%
  label tree/.style={
    for tree={tier/.option=level},
    level label/.style={
      before typesetting nodes={
        for nodewalk={current,tempcounta/.option=level,group={root,tree breadth-first},ancestors}{if={>OR={level}{tempcounta}}{before drawing tree={label me=##1}}{}},
      }
    },
    before drawing tree={
      tikz+={\coordinate (a) at (current bounding box.north);},
    },
  },
  label me/.style={tikz+={\node [anchor=base west] at (.parent |- a) {#1};}},
}
\renewcommand\theadfont{\bfseries}

\usepackage{tikz}
\newcolumntype{C}[1]{>{\centering\arraybackslash}p{#1}}
\newcommand{\sqht}{15px}
\arraycolsep=0pt

\newcommand{\ttt}[9]{\begin{array}{C{\sqht}|C{\sqht}|C{\sqht}}
  #1&#2&#3\\\hline
  #4&#5&#6\\\hline
  #7&#8&#9\\
\end{array}}

% \newcounter{num}
% \newcommand{\tictactoe}[1]
%     {
%     \begin{tikzpicture}[line width=1pt,scale=0.50]
%         \def\r{1.5mm}
%             \tikzset{
%                     circ/.pic={\draw circle (\r);},
%                     cross/.pic={\draw (-\r,-\r) -- (\r,\r) (-\r,\r) -- (\r,-\r);},
%                     opt/.pic={\draw[opacity=0.2] (-\r,-\r) -- (\r,\r) (-\r,\r) -- (\r,-\r);}
%                     }
            
%             % The grid
%             \foreach \i in {1,2} \draw (\i,0) -- (\i,3) (0,\i) -- (3,\i);
            
%             % Numbering the cells
%             \setcounter{num}{0}
%             \foreach \y in {0,...,2}
%                 \foreach \x in {0,...,2}
%                     {
%                     \coordinate (\thenum) at (\x+0.5,2-\y+0.5);
%                     %\node[opacity=0.5] at (\thenum) {\sffamily\thenum}; % Uncomment to see numbers in the cells
%                     \addtocounter{num}{1}
%                     }
                    
                    
%         \def\X{X} \def\x{x} \def\O{O} \def\n{n}
        
%         \foreach \l [count = \i from 0] in {#1}
%             {
%             \if\l\X \path (\i) pic{cross};
%             \else
%                 \if\l\O \path (\i) pic{circ};
%                 \else
%                     \if\l\x \path (\i) pic{opt};
%                     \else
%                         \if\l\n \node[opacity=0.5] at (\i) {\sffamily\i};
%                         \fi
%                     \fi
%                 \fi
%             \fi
%             }
%     \end{tikzpicture}
%     }

\setbeamercolor{background canvas}{bg=white}

\def\addsymbol#1#2#3{%
\expandafter\newcommand\csname #2\endcsname{#1}}

\newcommand{\refalgo}[1]{Algorithm \ref{#1}}
\newcommand{\reftbl}[1]{Table \ref{#1}}
\newcommand{\refsec}[1]{??}
\newcommand{\refthm}[1]{Theorem\,\ref{#1}}
\newcommand{\refcor}[1]{Corollary\,\ref{#1}}

\newcommand{\lsota}{state-of-the-art\xspace}  %lower-case version

\newcommand{\ie}{\emph{i.e.,}\xspace}
\newcommand{\eg}{\emph{e.g.,}\xspace}
\newcommand{\etc}{etc\xspace}

\newcommand{\ksi}{K+\sigma_n^2 I}
\newcommand{\inv}{^{\text{-}1}}

\newcommand{\nparam}{N_\mathrm{param}}
\newcommand{\gs}{{N_{\mathrm{gridsize}}}}
\newcommand{\nlvl}{N_\mathrm{lvl}}
\newcommand{\xinX}{x \in {\mathcal X}}

\newcommand{\matern}{Mat\'ern\xspace}
\newcommand\maternx [2]{Mat\'ern\,$\sfrac{#1}{#2}$\xspace}
\newcommand{\rbf}{RBF\xspace}
\newcommand{\kMat}{k_{\text{Mat}}}
\newcommand{\Rc}{\mathcal{R}}
\newcommand{\targetr}{\Rc_{\rm target}}

\newcommand{\parens}[1]{{\left(#1\right)}}
\newcommand{\braces}[1]{{\left\{#1\right\}}}
\newcommand{\sqparens}[1]{{\left[#1\right]}}
\newcommand{\bars}[1]{{\left|#1\right|}}
\newcommand{\dbars}[1]{{\left\|#1\right\|}}

\newcommand{\Xv}{\mathbf{X}}
\newcommand{\xv}{\mathbf{x}}
\newcommand{\Pv}{\mathbf{P}}
\newcommand{\xvtilde}{\widetilde{\mathbf{x}}}
\newcommand{\Fc}{\mathcal{F}}

% Special Method symbols
\newcommand{\graphoverlap}{\textsf{Graph\ Overlap}\xspace}
\newcommand{\graphnooverlap}{\textsf{Graph\ No-Overlap}\xspace}
\newcommand{\graphnooverlapl}[1]{{\textsf{Graph\ No-Overlap~(#1)}}\xspace}
\newcommand{\tree}{\textsf{Tree}\xspace}
\newcommand{\random}{\textsf{Random}\xspace}
\newcommand{\oracle}{\textsf{Oracle}\xspace}

\DeclareMathOperator*{\successrate}{Success-Rate}
\DeclareMathOperator*{\normalizedcost}{Normalized-Cost}
\DeclareMathOperator*{\fscore}{F_1score}
\DeclareMathOperator*{\precision}{Precision}
\DeclareMathOperator*{\recall}{Recall}
\DeclareMathOperator*{\Edge}{Edges}
\DeclareMathOperator{\Tr}{Tr}
\DeclareMathOperator{\sgn}{sgn}
\newcommand{\lgamma}{\mathlarger{\gamma}}
\newcommand{\union}{\cup}
\newcommand{\intersection}{\cap}

\newcommand{\fmin}{f_{\min}}
\newcommand{\fmax}{f_{\max}}

% \newtheorem{thm}{Theorem}[chapter]
% \newtheorem{cor}{Corollary}[chapter]

% Our Expts
\newcommand{\synthetic}{Synthetic1D\xspace}
\newcommand{\forrester}{Forrester1D\xspace}
\newcommand{\levy}{Levy1D\xspace}
\newcommand{\levyhard}{Levy-Hard1D\xspace}
\newcommand{\bohachevsky}{Bohachevsky2D\xspace}
\newcommand{\bohachevskyhard}{Bohachevsky-Hard2D\xspace}
\newcommand{\branin}{Branin2D\xspace}
\newcommand{\camelback}{Camelback2D\xspace}
\newcommand{\hartmann}{Hartmann6D\xspace}
\newcommand{\robotsmall}{Robot3D\xspace}
\newcommand{\robotlarge}{Robot4D\xspace}
\newcommand{\npsynthetic}{Synthetic1D-Online\xspace}
\newcommand{\npforrester}{Forrester1D-Online\xspace}
\newcommand{\npcamelback}{Camelback2D-Online\xspace}
\newcommand{\nplevy}{Levy1D-Online\xspace}
\newcommand{\nplevyhard}{Levy-Hard1D-Online\xspace}
\newcommand{\npbohachevsky}{Bohachevsky2D-Online\xspace}
\newcommand\levydef [2]{$C=#1,\eta^2=#2$\xspace}
\newcommand{\varforrester}{MaxVar-Forrester1D\xspace}
\newcommand{\varcamelback}{MaxVar-Camelback2D\xspace}
\newcommand{\varrobotsmall}{MaxVar-Robot3D\xspace}

% \newcommand{\random}{\textsf{Random}\xspace}
\newcommand{\noattack}{\textsf{No~Attack}\xspace}
\newcommand{\aggressivesubtraction}{\textsf{Aggressive~Subtraction}\xspace}
\newcommand{\subtraction}{\textsf{Subtraction}\xspace}
\newcommand{\clipping}{\textsf{Clipping}\xspace}
\newcommand{\subtractionrnd}{\textsf{Subtraction~Rnd}\xspace}
\newcommand{\shortsq}{\textsf{Sq}\xspace}
\newcommand{\shortrnd}{\textsf{Rnd}\xspace}
\newcommand{\subtractionsq}{\textsf{Subtraction~Sq}\xspace}
\newcommand{\bzero}{\boldsymbol{0}}

\newcommand{\rngadv}{\textsf{Rng-Adv}\xspace}
\newcommand{\boadv}{\textsf{BO-Adv}\xspace}

\newcommand{\rngbandit}{\textsf{Rng-Bandit}\xspace}
\newcommand{\optbandit}{\textsf{Optimal-Bandit}\xspace}
\newcommand{\ts}{\textsf{TS-Bandit}\xspace}
\newcommand{\ucb}{\textsf{UCB-Bandit}\xspace}
\newcommand{\egreedy}{\textsf{$\epsilon$-Greedy-Bandit}\xspace}

\newcommand{\tiling}{\textsf{Tiling}\xspace}
\newcommand{\fsft}{\textsf{FSFT}\xspace}
\newcommand{\sfsft}{\textsf{SFSFT}\xspace}

\newcommand{\anni}{\textsf{NNI}\xspace}
\newcommand{\lf}{\textsf{LF}\xspace}

\newcommand{\asq}{\textsf{Sq-Attack}\xspace}
\newcommand{\signopt}{\textsf{Sign-Opt}\xspace}

\newcommand{\gpucb}{GP-UCB\xspace}
\newcommand{\maxvar}{MaxVar\xspace}
\newcommand{\defense}{Defense\xspace}
\newcommand{\dynamic}{Dynamic\xspace}
\newcommand{\noprefit}{Online-Learned Kernel\xspace}

\newcommand{\Kv}{\mathbf{K}}
\newcommand{\kv}{\mathbf{k}}
\newcommand{\Iv}{\mathbf{I}}
\newcommand{\iv}{\mathbf{i}}
\newcommand{\yv}{\mathbf{y}}
\newcommand{\Yv}{\mathbf{Y}}

\DeclareMathOperator*{\argmax}{arg\,max}
\DeclareMathOperator*{\argmin}{arg\,min}

% Use \acs to reference them
% \addabbrev{AI}{Artificial Intelligence}
% \addabbrev{API}{Application Programming Interface}
% \addabbrev{ARD}{Automatic Relevance Determination}
% \addabbrev{BFGS}{Broyden-Fletcher-Goldfarb-Shanno algorithm}
% \addabbrev{BO}{Bayesian Optimization}
% \addabbrev{CNN}{Convolutional Neural Networks}
% \addabbrev{EI}{Expected Improvement}
% \addabbrev{FFT}{Fast Fourier Transform}
% \addabbrev{FMAB}{Federated Multi-armed Bandits}
% \addabbrev{GP}{Gaussian Process}
% \addabbrev{GP-UCB}{Gaussian Process with Upper Confidence Bound}
% \addabbrev{HDBO}{High-Dimensional Bayesian Optimization}
% \addabbrev{L-BFGS}{Limited-Memory Broyden-Fletcher-Goldfarb-Shanno algorithm}
% \addabbrev{LCB}{Lower Confidence Bound}
% \addabbrev{LF}{Low-frequency Fourier Transform}
% \addabbrev{MAB}{Multi-armed Bandits}
% \addabbrev{ML}{Machine Learning}
% \addabbrev{MLaaS}{Machine-Learning-as-a-Service}
% \addabbrev{MIP}{Mixed Integer Linear Programming}
% \addabbrev{NNI}{Nearest Neighbour Interpolation}
% \addabbrev{NN}{Neural Network}
% \addabbrev{RGB}{Red, Green, and Blue}
% \addabbrev{RBF}{Radial Basis Function}
% \addabbrev{REMBO}{Random EMbedding Bayesian Optimization}
% \addabbrev{RKHS}{Reproducing kernel Hilbert space}
% \addabbrev{TNC}{Truncated Newton algorithm}
% \addabbrev{UCB}{Upper Confidence Bound}
% \addabbrev{UF}{Union-Find}

\addsymbol{\mathbb{R}}{RR}{Set of real numbers}
\addsymbol{\mathbb{N}}{N}{Set of natural numbers}
\addsymbol{\mathbb{E}}{E}{Expectation}
\addsymbol{C}{ncycles}{Number of cycles}

\addsymbol{\mathrm{Bernoulli}}{ber}{Bernoulli distribution}
\addsymbol{\mathcal{X}}{X}{Domain}
\addsymbol{\mathcal{G}}{G}{Decomposition of functions}
\addsymbol{\phi}{featurespace}{Values in feature space}
\addsymbol{f}{f}{Target/Objective function $f:\X\rightarrow\RR$}
\addsymbol{N_\mathrm{dim}}{ndim}{Number of dimensions in $\X$}
\addsymbol{N_\mathrm{samples}}{nsamples}{Number of samples required}

\addsymbol{a_\mathrm{UCB}}{aUCB}{Upper Confidence Bound acquisition function}
\addsymbol{a_\mathrm{EI}}{aEI}{Expected Improvement acquisition function}

\addsymbol{\mathrm{cov}}{cov}{Covariance function}

\addsymbol{\mathbf{X}}{Xinput}{Input to the function $f$}
\addsymbol{\mathbf{y}}{youtput}{Output of the function $f$}

\addsymbol{\mathcal{D}}{D}{Observations/Samples, comprising of $\Xinput,\youtput$}
\addsymbol{x^*}{xopt}{Location on $f$ where it achieves global optimal value}
\addsymbol{f^*}{fopt}{Global Optima of $f$}
\addsymbol{x_{\min}}{xmin}{Location on $f$ where it achieves global minimum value}
\addsymbol{x_{\max}}{xmax}{Location on $f$ where it achieves global maximum value}
\addsymbol{\mathcal{GP}}{GP}{Gaussian Process modelling $f$ with mean $\mu$ and covariance $K$}

\newenvironment{pandoccrossrefsubfigures}
{
  \begin{figure}
}
{
  \end{figure}
}